% This template is public domain.
\documentclass{ltugboat}

%\usepackage{microtype}
\usepackage{graphicx}
\usepackage{ifpdf}
\ifpdf
\usepackage[breaklinks,hidelinks]{hyperref}
\else
\usepackage{url}
\fi

%%% Start of metadata %%%

\title{Accessible tables using `Tagged PDF'}

% repeat info for each author.
\author{Ross Moore}
%\address{Department of Mathematics and Statistics\\Macquarie University\\ Sydney, Australia}
\address{School of Mathematical and Physical Sciences\\Macquarie University\\ Sydney, Australia}
\netaddress{ross.moore@mq.edu.au}
\personalURL{https://researchers.mq.edu.au/en/persons/ross-moore}

%%% End of metadata %%%

\begin{document}
\maketitle

\begin{abstract}
Some basic requirements for Accessibility of tabular material are:
\begin{itemize}
\item 
 each cell, whether header or content, must have an attribute providing
 a unique ID for that cell;
\item 
 each data cell must specify the corresponding row and column headers
 that most directly provide the meaning of the information contained within the cell.
 This is done via a \textsf{Headers} attribute using the unique IDs for the header cells.
\end{itemize}
Header cells themselves may have other row or column headers; e.g., as a common header for a block of rows or columns.

Tagged PDF has the tagging and mechanisms to provide such attributes.
When the PDF is translated into HTML (using the  \textsf{ngPDF} online converter, say) this information
is recorded in the web-pages, to be available to Assistive Technologies.
In this talk we show several examples of tables specified using various packages, as in the \LaTeX\ Companion,
both in PDF and HTML web pages. 
A novel coding idea that allows this to be achieved will be presented.

\end{abstract}

\makesignature

\end{document}

\endinput

