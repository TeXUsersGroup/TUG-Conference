\documentclass{article}

\usepackage[T1]{fontenc}
\usepackage[utf8]{inputenc}
\usepackage[english,french]{babel}
\usepackage{jurabib}
\jurabibsetup{%
 citefull=first,super,ibidem,lookat,authorformat=allreversed,%
 titleformat=italic%
}

\newcommand{\ds}{\foreignlanguage{english}{Doc Savage}}
\newcommand{\web}{\foreignlanguage{english}{\emph{Web}}}

\renewcommand{\url}[1]{\relax}

\title{Essai du \foreignlanguage{english}{\emph{package} \texttt{jurabib}}}
\author{Jean-Michel Hufflen}
\date{Décembre 2014}

\begin{document}

\maketitle

Essai de quelques références \cite{kurland1977,lint2002} données çà et là dans
le document pédagogique déposé sur le \web\ \cite{h2006}. Au fait, ce document
pédagogique se continue en un second épisode \cite{h2006a} consacré au mode
mathématique de \TeX.

J'y ajouterais que quand j'étais adolescent, j'étais vraiment captivé par les
aventures de \ds. D'après certains, le meilleur épisode de la série est
\foreignlanguage{english}{\citefield{title}{robeson1974c}} \cite{robeson1974c},
où \ds\ est aux prises pour la seconde fois avec \foreignlanguage{english}{John
Sunlight}. J'ai personnellement beaucoup apprécié
\foreignlanguage{english}{\citefield{title}{robeson1968g}} \cite{robeson1968g}
et \foreignlanguage{english}{\citefield{title}{robeson1968j}}
\cite{robeson1968j}. \`{A} mon avis, la chute du dernier \cite{robeson1968j}
est vraiment éblouissante, même si je comprends que certains lui préfèrent
\foreignlanguage{english}{\citefield{title}{robeson1974c}} \cite{robeson1974c}.

\bibliography{l1-cmi}
\bibliographystyle{jurabib}

\end{document}
