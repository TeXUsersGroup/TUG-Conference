\documentclass{article}

\usepackage[T1]{fontenc}
\usepackage[latin1]{inputenc}
\usepackage[english,french]{babel}
\usepackage{url}

\newcommand{\ds}{\foreignlanguage{english}{Doc Savage}}
\newcommand{\web}{\foreignlanguage{english}{\emph{Web}}}

\title{Essai de bibliographies}
\author{Jean-Michel Hufflen}
\date{D�cembre 2014}

\begin{document}

\maketitle

Essai de quelques r�f�rences donn�es �� et l� dans le document p�dagogique
d�pos� sur le \web, c'est-�-dire \cite{h2006}. Entre autres ouvrages, on y
cite \cite{kurland1977,lint2002}. Au fait, ce document p�dagogique se continue
en un second �pisode consacr� au mode math�matique de \TeX\ \cite{h2006a}.

J'y ajouterais que quand j'�tais adolescent, j'�tais vraiment captiv� par les
aventures de \ds. D'apr�s certains, le meilleur �pisode de la s�rie est
\foreignlanguage{english}{\emph{The Devil Genghis}} \cite{robeson1974c}, o�
\ds\ est aux prises pour la seconde fois avec \foreignlanguage{english}{John
Sunlight}. Les �pisodes que j'ai personnellement beaucoup appr�ci�s sont
\cite{robeson1968g,robeson1968j}. \`{A} mon avis, la chute de
\cite{robeson1968j} est vraiment �blouissante, m�me si je comprends que
certains lui pr�f�rent \cite{robeson1974c}.

\bibliography{l1-cmi}
\bibliographystyle{alpha}

\end{document}
