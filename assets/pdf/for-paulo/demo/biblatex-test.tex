\documentclass{article}

\usepackage[T1]{fontenc}
\usepackage[utf8]{inputenc}
\usepackage[english,french]{babel}
\usepackage[backend=biber,bibstyle=numeric,citestyle=authoryear]{biblatex}
\addbibresource{l1-cmi.bib}

\newcommand{\ds}{\foreignlanguage{english}{Doc Savage}}

\title{Essai du \foreignlanguage{english}{\emph{package}} \texttt{biblatex}}
\author{Jean-Michel Hufflen}
\date{Décembre 2014}

\begin{document}

\maketitle

Essai de quelques références \cite{kurland1977,lint2002} données çà et là dans
le document pédagogique déposé sur le \foreignlanguage{english}{\emph{Web}}
\cite{h2006}. Au fait, ce document pédagogique se continue en un second épisode
\cite{h2006a} consacré au mode mathématique de \TeX.

J'y ajouterais que quand j'étais adolescent, j'étais vraiment captivé par les
aventures de \ds, par \foreignlanguage{english}{\citeauthor{robeson1968j}}.
D'après certains, le meilleur épisode de la série est
\foreignlanguage{english}{\citetitle{robeson1974c}} \cite{robeson1974c}, où
\ds\ est aux prises pour la seconde fois avec \foreignlanguage{english}{John
Sunlight}. J'ai personnellement beaucoup apprécié
\foreignlanguage{english}{\citetitle{robeson1968g}} et
\foreignlanguage{english}{\citetitle{robeson1968j}}
\cite{robeson1968g,robeson1968j}. \`{A} mon avis, la chute du dernier
\cite{robeson1968j} est vraiment éblouissante, même si je comprends que
certains lui préfèrent \foreignlanguage{english}{\citetitle{robeson1974c}}
\cite{robeson1974c}.

\printbibliography

\end{document}
